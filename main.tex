\documentclass{article}
\usepackage{chemfig}
\usepackage[backend=biber,style=numeric]{biblatex}
\addbibresource{ack.bib}

\title{Hypotenuse of a Biscuit}
\author{Michele Potter}

\begin{document}
\maketitle

\begin{abstract}
    We present conclusive but imaginary things.
\end{abstract}
\section{Introduction}
\textbf{I'm not logged in...}  - kat


We begin with lots of source citations such as \cite{pmid:10333997} and \cite{pmid:21394130}. There may be lots of accented names containing \'e and \"o. 

We then present an important theorem with many numbers and symbols.

Let L be an $\infty$ of biscuits. Let $\delta$ be an equivalence relation between silliness and something else.

\section{Proof}

Here we must prove.

$$\lim{x \to infty} f(x)$$.

$$\lim_{x \to a} \frac{f(x)-f(a)}{x-a},$$

\section{Conclusion}
And in the end, there will be cups of happiness for all!

\chemfig{
            % 1
      -[:42]N% 2
      -[:96]% 3
     =_[:24]N% 4
     -[:312]% 5
    =_[:240]% 6
               (
         -[:168]\phantom{N}% -> 2
               )
     -[:300]% 7
               (
         =[:240]O% 8
               )
           -N% 9
               (
         -[:300]% 14
               )
      -[:60]% 10
               (
               =O% 11
               )
     -[:120]N% 12
               (
         -[:180]% -> 5
               )
      -[:60]% 13
}






\printbibliography
\end{document}